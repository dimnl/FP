\documentclass{beamer}
\usepackage[utf8]{inputenc}

\usepackage{color}
\usepackage{listings}

%% Setup
\lstset{
%	language=Haskell,
	basicstyle=\small\sffamily,
%	numbers=left,
%	numberstyle=\tiny,
%	frame=tb,
	tabsize=4,
	columns=fixed,
	showstringspaces=false,
	showtabs=false,
	keepspaces,
	commentstyle=\color{red},
	keywordstyle=\color{blue}
}


% \usepackage{roboto}

\usetheme{metropolis} % Hannover is also nice
%\usecolortheme{dove}

%------------------------------------------------------------
% Information to add to title page
\title{Functional Programming}
\subtitle{Second project: Idris}
\author{D.C.M. Hoogeveen}
\institute{University of Twente}
\date{July 3, 2019}

%------------------------------------------------------------
%The next block of commands puts the table of contents at the 
%beginning of each section and highlights the current section:

% \AtBeginSection[]
% {
%   \begin{frame}
%     \frametitle{Table of Contents}
%     \tableofcontents[currentsection]
%   \end{frame}
% }
%------------------------------------------------------------

\begin{document}

\frame{\titlepage} % Title page

% Table of contents
\begin{frame}
\frametitle{Table of Contents}
\tableofcontents
\end{frame}

\section{Idris in general}

\begin{frame} % First slide
\frametitle{Idris characteristics}

\begin{itemize}
	\item \textbf{Functional programming} language
	\item Haskell-inspired syntax
	\item \textbf{Strict} evaluation order
	\item Installable using Cabal
	\item Pac-man Complete % as its inventor Edwin brady says: able to build pac-man with it
	\item \textbf{Compiled} (via e.g. C and Javascript); 
		\begin{itemize}
			\item Optimisations possible (e.g. aggresive erasure and inlining) % such as aggresive erasure, inlining and partial evaluation
		\end{itemize}
		
	\item \textbf{Dependent type} system
\end{itemize}
\end{frame}

\begin{frame}[fragile]
\frametitle{Dependent types: what?}


\begin{example}
	Looking at following type definition, what can we infer?
\end{example}

\begin{lstlisting}
foo : Vect len elem -> Vect len elem
\end{lstlisting}

\pause

\begin{block}{Inferred}
	Same length output as input and with the same type element. 
\end{block}

\pause

\begin{alertblock}{Conclusion}
	Dependent types allow us to define output dependent on the input.
\end{alertblock}

\end{frame}

\begin{frame}
\frametitle{Dependent type: why?}
	Useful for:
	\begin{itemize}
		\item \textbf{Checking} intended properties
		\item \textbf{Guiding} programmer
		\item Building \textbf{generic} libraries
	\end{itemize}

\end{frame}

\begin{frame}
\frametitle{Type Drive Development}
\begin{alertblock}{Type}
	Write the type definition 
\end{alertblock}	

\begin{alertblock}{Define}
	Create a stub implementation; can also be a hole: \textit{?x}
\end{alertblock}

\begin{alertblock}{Refine}
	Improve/complete implementation
\end{alertblock}

\end{frame}

\section{Idris in practice}


\begin{frame}[fragile]
\frametitle{Basics}
\begin{example}
Take first element of a vector:
\end{example}

\begin{lstlisting}
myhead : Vect (S len) elem -> elem
myhead (x :: xs) = x
\end{lstlisting}

\begin{block}{Advantage}
	Normally, a check is needed to prevent an error when a vector with zero length is used, however this is already defined using (S len)!
\end{block}


\end{frame}

\begin{frame}[fragile]
\frametitle{Idxsize}
\begin{example}
Count number of values that are True within a vector recursively 
\end{example}

\pause
\begin{lstlisting}
idxsize : Vect n Bool -> Nat
\end{lstlisting} \pause
\begin{lstlisting}
idxsize [] = 0
\end{lstlisting} \pause
\begin{lstlisting}
idxsize (True :: xs)  = 1 + idxsize xs
idxsize (False :: xs) = idxsize xs
\end{lstlisting}

\end{frame}

\begin{frame}[fragile]
\frametitle{Usage of idxsize as type}

\begin{example}
Combine two vectors by only returning element of second vector if element of first vector is True
\end{example}
Defined type definition:
\begin{lstlisting}
get : (xs : Vect n Bool) -> Vect n a -> Vect ? a
\end{lstlisting}

Expected use:
\begin{lstlisting}
get [True,False,True] [1,2,3] == [1,3]
\end{lstlisting}

\textbf{What to fill in for '?'? }


\end{frame}

\begin{frame}[fragile]
\frametitle{Usage of idxsize as type}
\begin{example}
\end{example}
\begin{lstlisting}
get : (xs : Vect n Bool) -> Vect n a -> Vect ? a
\end{lstlisting}

Expecting a Nat (natural number type) of amount of values in first vector whose value is True.
\pause
\begin{block}{Recall idxsize}
\end{block}
\begin{lstlisting}
idxsize : Vect n Bool -> Nat
\end{lstlisting}

This is exactly the type needed for '?'! 
\pause
\begin{alertblock}{Solution}
	? = idxsize xs
\end{alertblock}
\end{frame}

\section{Questions}

\end{document}
